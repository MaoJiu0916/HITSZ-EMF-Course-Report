\documentclass[a4paper,12pt]{article}

\usepackage{ctex}
\usepackage{tikz}
\usepackage{geometry}
\usepackage{emoji}
\geometry{left=3.18cm,right=3.18cm,top=2.54cm,bottom=2.54cm}
\title{2021秋《电磁场》课程EE2003课程报告}
\author{姓名:猫九 学号:11111111  19级搬砖类}

\begin{document}
\maketitle
\begin{center}
	\begin{tikzpicture}
		\draw[dashed][very thick] (0,0)--(14.64,0);
	\end{tikzpicture}	
\end{center}
\noindent \textbf{例1-13}一点电荷$q$放置在内表面半径为$b$,厚度为$c$的导体球壳内,点电荷与球心的距离为$a$。分别求在球壳接地和不接地的两种情况下点电荷所受的力。
 
\noindent \textbf{解:} 

\noindent (1) 当球壳接地时,如图所示:

\begin{center}
		\begin{tikzpicture}	
		\draw[dashed] (0,0) circle [radius=1.5];
		\draw[thick] (0,0) circle [radius=2.5];
		\draw[dash dot dot] (-4,0) -- (4,0);
		\draw[-latex][very thick] (0,0) node[below] {$O$} -- node[above] {$a$}  (-1,0) node[above] {$q$} ;
		\fill (-1,0) circle[radius=2pt];  
		\draw[-latex][very thick] (0,0) -- node[above left] {$b$} (1.061,1.061); 
		\draw[latex-latex][very thick] (1.061,1.061) -- node[above left] {$b$} (1.768,1.768); 
		\coordinate (S1) at (-1.768,-1.768); 
		\draw[thick] (S1) -- (-1.768,-3.2328);
		\draw[thick] (-2.068,-3.2328) -- (-1.468,-3.2328);
		\draw[thick] (-1.968,-3.3328) -- (-1.568,-3.3328);
		\draw[thick] (-1.868,-3.4328) -- (-1.668,-3.4328);		
	\end{tikzpicture}
\end{center}

\noindent 由于球壳接地,所以球壳外的电场$E=0$ ,求点电荷所受的力时,可以使用镜像法。由教材42页的例题可知,点电荷附近接地导体球的影响,可以由位于距离球心$b$处的镜像电荷$(-q^{'})$的来表示。

\noindent 在本题中,只需要求球壳内的电场强度。球壳的厚度$c$不产生影响。在球壳外距离球心$$d=\frac{R^2}{a}=\frac{b^2}{a}$$
处放置一个电荷,如图所示

$$
q^{\prime}=-\frac{d}{b}q=-\frac{b}{a}q
$$

\begin{center}
	\begin{tikzpicture}	
		\draw[dashed] (0,0) circle [radius=1.5];
		\draw[thick] (0,0) circle [radius=2.5];
		\draw[dash dot dot] (-4,0) -- (4,0);
		\draw[-latex][very thick] (0,0) node[below] {$O$} -- node[above] {$a$}  (-1,0) node[above] {$q$} ;
		\fill (-1,0) circle[radius=2pt];  
		\draw[-latex][very thick] (0,0) -- node[above left] {$b$} (1.061,1.061); 
		\draw[latex-latex][very thick] (1.061,1.061) -- node[above left] {$b$} (1.768,1.768); 
		\coordinate (S1) at (-1.768,-1.768); 
		\draw[thick] (S1) -- (-1.768,-3.2328);
		\draw[thick] (-2.068,-3.2328) -- (-1.468,-3.2328);
		\draw[thick] (-1.968,-3.3328) -- (-1.568,-3.3328);
		\draw[thick] (-1.868,-3.4328) -- (-1.668,-3.4328);
		\draw[red]	(0,0) -- (0,-0.8);	
		\fill (3,0) circle[radius=2pt]; 
		\draw[red] (3,0) node[above][black] {$q^{'}$} -- (3,-0.8);
		\draw[latex-latex][very thick][red] (0,-0.5) -- node[below][black] {$d$} (3,-0.5); 
	\end{tikzpicture}
\end{center}

\noindent 该电荷位于球心和点电荷$q$的连线上。它对点电荷$q$的作用力就是球壳对点电荷$q$的作用力,大小为
$$
F=\frac{q^{\prime}q}{4\pi \varepsilon _0\left( d-a \right) ^2}=-\frac{abq^2}{4\pi \varepsilon _0\left( b^2-a^2 \right) ^2}
$$
\noindent 其中负号表示两个电荷之间是异种电荷,是吸引力。

~\\
\noindent(2) 对于球壳不接地的情况,球壳外表面产生的感应电荷在球壳内不产生电场,球壳表面的感应电荷也可以用$(-q^{'})$来代替,球壳内电场强度的分布不变。点电荷$q$受到的力和球壳接地时受到的力相同。

~\\
\noindent \textbf{本题目考查知识点:}
\begin{enumerate}
	\item[1.] \textbf{镜像法}求导体球面问题,教材P42,公式(1-78)
	\item[2.] 静电场的\textbf{唯一性定理},教材P28
	\item[3.] \textbf{库仑定律}	
\end{enumerate}

~\\


\noindent \textbf{对本学期《 电磁场》课程及授课教师的意见和建议:}

由老师上的《 电磁场》课程非常好,由老师讲很课细致,而且上课的过程中还讲述与课程相关的故事和自己的经历,双语教学也让我们体验到了不一样的上课方式,从电磁场这门课程中不仅学到了专业知识,也领会到了其他的方方面面。

由老师yyds!\emoji{+1}

唯一遗憾的是最后几节课没有能在线下听由老师讲课,以及之后不确定能不能在深圳再见到由老师。

对于课程的建议:希望教务部在排课程的时候不要一周排三节课,感觉没开学几天就结课了,\emoji{joy},而且周五上两节电磁场课程比较难顶。
	
\end{document}